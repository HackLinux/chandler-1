\section{wxTreeCtrl overview}\label{wxtreectrloverview}

Classes: \helpref{wxTreeCtrl}{wxtreectrl}, \helpref{wxImageList}{wximagelist}

The tree control displays its items in a tree like structure. Each item has its
own (optional) icon and a label. An item may be either collapsed (meaning that
its children are not visible) or expanded (meaning that its children are
shown). Each item in the tree is identified by its {\it itemId} which is of
opaque data type {\it wxTreeItemId}. You can test whether an item is valid
by calling wxTreeItemId::IsOk.

The items text and image may be retrieved and changed with 
\helpref{GetItemText}{wxtreectrlgetitemtext}/\helpref{SetItemText}{wxtreectrlsetitemtext} 
and 
\helpref{GetItemImage}{wxtreectrlgetitemimage}/\helpref{SetItemImage}{wxtreectrlsetitemimage}.
In fact, an item may even have two images associated with it: the normal one
and another one for selected state which is set/retrieved with 
\helpref{SetItemSelectedImage}{wxtreectrlsetitemselectedimage}/\helpref{GetItemSelectedImage}{wxtreectrlgetitemselectedimage} 
functions, but this functionality might be unavailable on some platforms.

Tree items have several attributes: an item may be selected or not, visible or
not, bold or not. It may also be expanded or collapsed. All these attributes
may be retrieved with the corresponding functions: 
\helpref{IsSelected}{wxtreectrlisselected}, 
\helpref{IsVisible}{wxtreectrlisvisible}, \helpref{IsBold}{wxtreectrlisbold} 
and \helpref{IsExpanded}{wxtreectrlisexpanded}. Only one item at a time may be
selected, selecting another one (with 
\helpref{SelectItem}{wxtreectrlselectitem}) automatically unselects the
previously selected one.

In addition to its icon and label, a user-specific data structure may be associated
with all tree items. If you wish to do it, you should derive a class from {\it
wxTreeItemData} which is a very simple class having only one function {\it
GetId()} which returns the id of the item this data is associated with. This
data will be freed by the control itself when the associated item is deleted
(all items are deleted when the control is destroyed), so you shouldn't delete
it yourself (if you do it, you should call 
\helpref{SetItemData(NULL)}{wxtreectrlsetitemdata} to prevent the tree from
deleting the pointer second time). The associated data may be retrieved with 
\helpref{GetItemData()}{wxtreectrlgetitemdata} function.

Working with trees is relatively straightforward if all the items are added to
the tree at the moment of its creation. However, for large trees it may be
very inefficient. To improve the performance you may want to delay adding the
items to the tree until the branch containing the items is expanded: so, in the
beginning, only the root item is created (with 
\helpref{AddRoot}{wxtreectrladdroot}). Other items are added when
EVT\_TREE\_ITEM\_EXPANDING event is received: then all items lying immediately
under the item being expanded should be added, but, of course, only when this
event is received for the first time for this item - otherwise, the items would
be added twice if the user expands/collapses/re-expands the branch.

The tree control provides functions for enumerating its items. There are 3
groups of enumeration functions: for the children of a given item, for the
sibling of the given item and for the visible items (those which are currently
shown to the user: an item may be invisible either because its branch is
collapsed or because it is scrolled out of view). Child enumeration functions
require the caller to give them a {\it cookie} parameter: it is a number which
is opaque to the caller but is used by the tree control itself to allow
multiple enumerations to run simultaneously (this is explicitly allowed). The
only thing to remember is that the {\it cookie} passed to 
\helpref{GetFirstChild}{wxtreectrlgetfirstchild} and to 
\helpref{GetNextChild}{wxtreectrlgetnextchild} should be the same variable (and
that nothing should be done with it by the user code).

Among other features of the tree control are: item sorting with 
\helpref{SortChildren}{wxtreectrlsortchildren} which uses the user-defined comparison
function \helpref{OnCompareItems}{wxtreectrloncompareitems} (by default the
comparison is the alphabetic comparison of tree labels), hit testing
(determining to which portion of the control the given point belongs, useful
for implementing drag-and-drop in the tree) with 
\helpref{HitTest}{wxtreectrlhittest} and editing of the tree item labels in
place (see \helpref{EditLabel}{wxtreectrleditlabel}).

Finally, the tree control has a keyboard interface: the cursor navigation (arrow) keys
may be used to change the current selection. <HOME> and <END> are used to go to
the first/last sibling of the current item. '+', '-' and '*' expand, collapse
and toggle the current branch. Note, however, that <DEL> and <INS> keys do
nothing by default, but it is usual to associate them with deleting item from
a tree and inserting a new one into it.

